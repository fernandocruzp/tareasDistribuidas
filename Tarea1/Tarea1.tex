\documentclass{article}
\usepackage[utf8]{inputenc}
\usepackage{amsmath,amsfonts,amssymb}
\usepackage{geometry}
\usepackage{color}
\usepackage[spanish]{babel}
\geometry{letterpaper, margin=1in}
\usepackage{listings}
\lstset{
  language=Python,
  basicstyle=\small\ttfamily,
  keywordstyle=\color{blue},
  commentstyle=\color{green},
  stringstyle=\color{red},
  breaklines=true,
  showstringspaces=false,
  numbers=left,
  numberstyle=\tiny,
  frame=lines
}
\usepackage{setspace} \doublespacing

\title{Tarea 1}
\author{Fernando Cruz Pineda
  \\
  
}
\date{\today}
\begin{document}
\maketitle
\section*{Ejercicio 1}
¿Cuál es la diferencia entre el cómputo concurrente, el cómputo paralelo y el cómputo distribuido?

\section*{Ejercicio 2}
¿Por qué no hay un único modelo de cómputo distribuido?

\section*{Ejercicio 3}
¿Por qué el cómputo distribuido se relaciona a menudo con el término Big Data?

\section*{Ejercicio 4}
Explica por qué es posible tener paralelismo sin concurrencia y concurrencia sin paralelismo.

\section*{Ejercicio 5}
¿Cuáles son las diferencias entre un sistema síncrono, un sistema asíncrono, un algoritmo síncrono y un algoritmo asíncrono?

\section*{Ejercicio 6}
¿El internet es un sistema síncrono o asíncrono? Justifica tu respuesta.

\section*{Ejercicio 7}
¿Cómo se define la complejidad en tiempo para los algoritmos síncronos?

\section*{Ejercicio 8}
Explica con tus propias palabras el modelo LOCAL y el modelo CONGEST.

\section*{Ejercicio 9}
¿Qué es la contención de nodos?

\section*{Ejercicio 10}
Demuestra el siguiente lema:

\begin{lemma}
Una gráfica no dirigida es un árbol si y solo si existe exactamente un camino simple entre cada par de vértices.
\end{lemma}

\section*{Ejercicio 11}
Explica con tus propias palabras el funcionamiento de un algoritmo distribuido que calcule la distancia entre la raíz de una gráfica y el nodo que se está visitando.

\section*{Ejercicio 12}
Para el algoritmo propuesto en el ejercicio anterior, argumenta cuántos mensajes es necesario enviar para diseminar el mensaje y cuál es el tamaño de cada mensaje enviado.

\end{document}
